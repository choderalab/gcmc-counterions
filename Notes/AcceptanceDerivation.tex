% !TEX TS-program = pdflatex
% !TEX encoding = UTF-8 Unicode

% This is a simple template for a LaTeX document using the "article" class.
% See "book", "report", "letter" for other types of document.

\documentclass[11pt]{article} % use larger type; default would be 10pt

\usepackage[utf8]{inputenc} % set input encoding (not needed with XeLaTeX)

%%% Examples of Article customizations
% These packages are optional, depending whether you want the features they provide.
% See the LaTeX Companion or other references for full information.

%%% PAGE DIMENSIONS
\usepackage{geometry} % to change the page dimensions
\geometry{a4paper} % or letterpaper (US) or a5paper or....
\geometry{margin=1in} % for example, change the margins to 2 inches all round
% \geometry{landscape} % set up the page for landscape
%   read geometry.pdf for detailed page layout information

\usepackage{graphicx} % support the \includegraphics command and options

% \usepackage[parfill]{parskip} % Activate to begin paragraphs with an empty line rather than an indent

%%% PACKAGES
\usepackage{booktabs} % for much better looking tables
\usepackage{array} % for better arrays (eg matrices) in maths
\usepackage{paralist} % very flexible & customisable lists (eg. enumerate/itemize, etc.)
\usepackage{verbatim} % adds environment for commenting out blocks of text & for better verbatim
\usepackage{subfig} % make it possible to include more than one captioned figure/table in a single float
\usepackage{amsmath}
% These packages are all incorporated in the memoir class to one degree or another...

%%% HEADERS & FOOTERS
\usepackage{fancyhdr} % This should be set AFTER setting up the page geometry
\pagestyle{fancy} % options: empty , plain , fancy
\renewcommand{\headrulewidth}{0pt} % customise the layout...
\lhead{}\chead{}\rhead{}
\lfoot{}\cfoot{\thepage}\rfoot{}

%%% SECTION TITLE APPEARANCE
\usepackage{sectsty}
\allsectionsfont{\sffamily\mdseries\upshape} % (See the fntguide.pdf for font help)
% (This matches ConTeXt defaults)

%%% ToC (table of contents) APPEARANCE
\usepackage[nottoc,notlof,notlot]{tocbibind} % Put the bibliography in the ToC
\usepackage[titles,subfigure]{tocloft} % Alter the style of the Table of Contents
\renewcommand{\cftsecfont}{\rmfamily\mdseries\upshape}
\renewcommand{\cftsecpagefont}{\rmfamily\mdseries\upshape} % No bold!

%%% END Article customizations

%%% The "real" document content comes below...

\title{The `constant salt ensemble'}
\author{Gregory Ross}
%\date{} % Activate to display a given date or no date (if empty),
         % otherwise the current date is printed 


\begin{document}
\maketitle

\begin{abstract}
Rough investigation of the ensemble and acceptance test used for constant counterion Monte Carlo simulations as initiated by John Chodera. 
\end{abstract}

\section{The model}

The aim is to simulate a region that exchanges molecules with a bath of bulk water with a fixed concentration of NaCl. The simulated region and bath are labeled $a$ and $b$, with volumes $V_a$ and $V_b$ respectively. Like in the Gibbs ensemble, the interaction energy between the two compartments is zero. Both compartments will have a neutral charge. The total volume, $V_a + V_b$, is constant, and we shall initially assume that subvolumes are also constant, although this can be relaxed later. The total number of molecules, $N$, is constant, and is decomposed 

\begin{align}
N = n_w^a + n_w^b + n_{Na}^a + n_{Na}^b + n_{Cl}^a + n_{Cl}^b,
\end{align}

where $n_w^a$ denotes the number of water molecules in compartment $a$, $n_w^b$ the number of water molecules in compartment $b$, and similarly for sodium and chlorine. The probability density for observing compartment $a$ and $b$ with a configuration $(\mathbf{x}_a,\mathbf{x}_b)$ and set of compartment species numbers $\mathbf{n}=  (n_w^a, n_w^b, n_{Na}^a, n_{Na}^b, n_{Cl}^a, n_{Cl}^b)$ is given by

\begin{align}
% \pi(\mathbf{x}_a,\mathbf{x}_b,\mathbf{n}|\beta,V_a,V_b) &\propto \frac{V_a^{n_w^a}}{n_w^a!}\frac{V_b^{n_w^b}}{n_w^b!}\frac{V_a^{n_{Na}^a}}{n_{Na}^a!}\frac{V_b^{n_{Na}^b}}{n_{Na}^b!}\frac{V_a^{n_{Cl}^a}}{n_{Cl}^a!}\frac{V_b^{n_{Cl}^b}}{n_{Cl}^b!} \exp\big(-\beta U(\mathbf{x}_a,\mathbf{x}_b)\big), \\
\pi(\mathbf{x}_a,\mathbf{x}_b,\mathbf{n}|\beta,V_a,V_b) \propto  \prod_{i,j} \frac{V_i^{n_j^i}}{n^i_j! \, \Lambda^{3n^i_j}}\, e^{-u(\mathbf{x}_a,\mathbf{x}_b)}
\end{align}

where $i \in \{a,b\}$, $j \in \{w, Na, Cl\}$, and $u(\mathbf{x}_a,\mathbf{x}_b)$ is the dimensionless interaction energy.

\section{Acceptance test derivation}
To avoid the creation and annihilation of particles and to keep the simulated region (compartment $a$) neutral, exchange moves involve simultaneously swapping two water molecules for Na and Cl. The acceptance probability will be derived by evaluating the acceptance probability for each stage of the swap individually, then amalgamating them all in a single move and acceptance probability. 

For sampling exchanges with the Metropolis-Hastings algorithm with symmetric transition probabilities, the acceptance probability for removing one particle labeled $i$ (i.e. Na or Cl) from compartment $b$ and adding it to $a$, but maintaining a fixed configuration (i.e. an identity exchange) is given by

\begin{align}
A(n_i^a,n_i^b \rightarrow n_i^a + 1, n_i^b-1;\mathbf{x}_a,\mathbf{x}_b) &= \frac{\pi(n_i^a + 1, n_i^b-1)}{\pi(n_i^a,n_i^b)} \notag\\
&= \frac{V_a}{V_b} \frac{n^b_i}{n^a_i + 1} \, e^{-\Delta u_i(\mathbf{x}_a) - \Delta u_i(\mathbf{x}_b)}.
\end{align}

The acceptance probability to exchange two water molecules from $a$ to $b$ is composed of two annihilation from $a$ and two creations in $b$. Because the interaction energy between compartments $a$ and $b$ is zero, $A(n_w^a,n_w^b \rightarrow n_w^a - 2,n_w^b + 2) = A(n_w^a \rightarrow n_w^a - 2)A(n_w^b \rightarrow n_w^b + 2)$. For the first term on the LHS, we have

\begin{align}
A(n_w^a \rightarrow n_w^a - 2;\mathbf{x}_a,\mathbf{x}_b) & = \frac{\pi(n_w^a - 2)}{\pi(n_w^a)} \notag\\
&= \frac{n_w^a(n_w^a - 1)(\Lambda_w)^6 }{(V_a)^2} e^{-\Delta u_w(\mathbf{x}_a)},
\end{align}

and for the second term of the LHS we have
\begin{align}
A(n_w^b \rightarrow n_w^b + 2;\mathbf{x}_a,\mathbf{x}_b) &= \frac{\pi(n_w^b + 2)}{\pi(n_w^b)} \notag\\
&= \frac{(V_b)^2}{(n_w^b +1)(n_w^b +2)(\Lambda_w)^6} \, e^{-\Delta u_w(\mathbf{x}_b)},
\end{align}

so that

\begin{align}
A(n_w^a,n_w^b \rightarrow n_w^a - 2,n_w^b + 2;\mathbf{x}_a,\mathbf{x}_b) = \left(\frac{V_b}{V_a}\right)^2 \frac{n_w^a(n_w^a - 1)}{(n_w^b +1)(n_w^b +2)} \, e^{- \Delta u_w(\mathbf{x}_a)- \Delta u_w(\mathbf{x}_b)}.
\end{align}

Therefore, the total acceptance probability for removing one atom of $Na$ and one atom of $Cl$ from compartment $b$, and adding them to compartment $a$, whilst simultaneously removing two water molecules from $a$ and adding them to $b$ is given by

\begin{align}
A(\text{add NaCl to} \,\, a;\mathbf{x}_a,\mathbf{x}_b) =  \frac{n_w^a(n_w^a - 1)}{(n_w^b +1)(n_w^b +2)}\frac{n^b_{Na}n^b_{Cl}}{(n^a_{Na} + 1)(n^a_{Cl} + 1)} \, e^{- \Delta u(\mathbf{x}_a)- \Delta u(\mathbf{x}_b)},
\end{align}

where $u(\mathbf{x}_a)$ and $\Delta u(\mathbf{x}_b)$ are the total changes in reduced potential energy between compartments $a$ and $b$. Similarly, the acceptance probability to remove NaCl from compartment $a$ and replace with two water molecules from compartment $b$ is given by

\begin{align}
A(\text{remove NaCl from} \,\, a;\mathbf{x}_a,\mathbf{x}_b) =  \frac{n_{Na}^a n^a_{Cl}}{(n_{Na}^b +1)(n_{Cl}^b+1)}\frac{n^b_{w}(n^b_{w}-1)}{(n^a_{w} + 1)(n^a_{w} + 2)} \, e^{- \Delta u(\mathbf{x}_a)- \Delta u(\mathbf{x}_b)},
\end{align}


\subsection{When the bath is very large and unobserved}
We consider the case when compartment $b$ (the bulk water and salt bath) becomes macroscopic in size. Collecting all the terms that correspond to $b$ and taking the thermodynamic limits $V_b \rightarrow \infty$, $n_i^b \rightarrow \infty$, and $n^b_i/V_b \rightarrow \rho_i$ $\forall \, i$, we find that

\begin{align}
A(\text{add NaCl to} \,\, a;\mathbf{x}_a,\mathbf{x}_b) &\rightarrow \frac{\rho_{Na}^b \rho_{Cl}^b}{(\rho_w^b)^2}\frac{n_w^a(n_w^a - 1)}{(n^a_{Na} + 1)(n^a_{Cl} + 1)}e^{\Delta u(\mathbf{x}_a)- \Delta u(\mathbf{x}_b)} \notag\\
&=\left(\frac{\rho_{NaCl}^b}{\rho_w^b}\right)^2\frac{n_w^a(n_w^a - 1)}{(n^a_{Na} + 1)(n^a_{Cl} + 1)}e^{\Delta u(\mathbf{x}_a)- \Delta u(\mathbf{x}_b)}.
\end{align}

where the second line follows from the stoichiometry of neutral salt. Finally, we require that the macroscopic bath is unobserved, so will use the acceptance probability averaged over all configurations of $x_b$

\begin{align}
A(\text{add NaCl to} \,\, a;\mathbf{x}_a)  &= \int_{V_b} \pi(\mathbf{x}_b)\, A(\text{add NaCl to} \,\, a;\mathbf{x}_a,\mathbf{x}_b) \, d\mathbf{x}_b \notag\\
&= \left(\frac{\rho_{NaCl}^b}{\rho_w^b}\right)^2\frac{n_w^a(n_w^a - 1)}{(n^a_{Na} + 1)(n^a_{Cl} + 1)}\exp(-\Delta u(\mathbf{x}_a))\langle \exp-\Delta u(\mathbf{x}_b)) \rangle_b \notag\\
&= \left(\frac{\rho_{NaCl}^b}{\rho_w^b}\right)^2\frac{n_w^a(n_w^a - 1)}{(n^a_{Na} + 1)(n^a_{Cl} + 1)} \, e^{-\Delta u(\mathbf{x}_a)}\,\big\langle e^{-\Delta u(\mathbf{x}_b)} \big\rangle_b \notag\\ 
&= \frac{n_w^a(n_w^a - 1)}{(n^a_{Na} + 1)(n^a_{Cl} + 1)} \, e^{-\Delta u(\mathbf{x}_a)} \, \left[ \left(\frac{\rho^b_{NaCl}}{\rho_w^b}\right)^2 \, e^{-\Delta f} \right],
\end{align}

where $\Delta f$ is the dimensionless free energy to decouple NaCl from bulk water at a given concentration minus the free energy to couple two water molecules from bulk water. Similarly, we find

\begin{align}
A(\text{remove NaCl from} \,\, a;\mathbf{x}_a,\mathbf{x}_b) = \frac{n_{Na}^a n_{Cl}^a}{(n_{w}^a+1)(n_{w}^a+2)}  \left [\left(\frac{\rho_{w}^b}{\rho_{NaCl}^b}\right)^2e^{+\Delta f} \right].
\end{align}

Note that the terms within the square brackets are completely determined by the thermodynamic bath. At low $NaCl$ concentrations, $\Delta f$ could be evaluated via alchemical free energy calculations although in general, it is a free parameter that requires calibration. In contrast, $\rho_{NaCl}$ and $\rho_{w}$ are set by the user. 

The acceptance test above is corresponds to the semi grand canonical ensemble. The identity exchange has no dependence on $V_a$, so volume can fluctuate during the simulation.

\end{document}
